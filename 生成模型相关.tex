\documentclass[12pt, a4paper, oneside]{ctexart}
\usepackage{amsmath, amsthm, amssymb, graphicx,color}
\usepackage[bookmarks=true, colorlinks, citecolor=blue, linkcolor=black]{hyperref}

% 导言区

\title{生成模型介绍}
\author{Shwj}
\date{\today}

\begin{document}
% \begin{sloppypar}

\maketitle

\section{目录与计划}

\begin{enumerate} 
    \item First Learnfk.com
    \item Hi Learnfk
        \begin{itemize}
            \item 介绍一下信息论的基础内容 KL散度之类的
            \item 从EM算法开始讲起 讲一下EM算法
            % \item https://www.bing.com/search?q=latex不等于&cvid=96253826ce5f499e8171ef88d7929320&aqs=edge..69i57j0.4410j0j4&FORM=ANAB01&PC=U531
            \item https://blog.csdn.net/junruitian/article/details/104406067
        \end{itemize}
    \item last LEARNFK
\end{enumerate}

\section{数学基础}
\subsection{KL散度和信息论}
\par
KL散度起源于信息论。信息论的主要目标是量化数据中有多少信息。信息论中最重要的指标称为熵,通常表示为H。概率分布的熵的定义是$$H=-\sum_{i=1}^{N}{p(x_i)\log p(x_i)}$$
\par
我们可以把熵解释为“我们编码信息所需要的最小比特数”,熵的关键在于,只要知道所需位数的理论下限,我们就可以准确地量化数据中有多少信息。
\par
KL 散度,是一个用来衡量两个概率分布的相似性的一个度量指标。我们知道,\textbf{现实世界里的任何观察都可以看成表示成信息和数据},一般来说,我们无法获取数据的总体,我们只能拿到数据的部分样本,根据数据的部分样本,我们会对数据的整体做一个近似的估计,而数据整体本身有一个真实的分布(我们可能永远无法知道)。
那么\textbf{近似估计的概率分布和数据整体真实的概率分布的相似度},或者说差异程度,可以用 KL 散度来表示。对于两个离散性随机变量$x\sim p(x)$和$x\sim q(x)$Kullback-Leibler divergence定义为$$D_{KL}(p||q)=\sum_{i=1}^{N}{p(x_i)(\log p(x_i)-\log q(x_i))}$$ 
而当p和q为连续随机变量时$$D_{KL}(p||q)=\int_{-\infty}^{+\infty}{p(x)\ln \frac{p(x)}{q(x)}}$$
\par
KL散度不满足对称性,即$D_{KL}(p||q)\neq D_{KL}(q||p)$所以它不能作为一种度量函数。KL散度可以衡量使用基于Q的分布来编码服从P的分布的样本所需的额外的平均比特数。典型情况下,P表示数据的真实分布,Q表示数据的理论分布、估计的模型分布、或P的近似分布。\footnote{来自Wikipedia}
\par
KL散度很重要的性质是$D_{KL}(p||q)\geq 0$ 因为“使用一种编码方式编码另一套符号总会有损失”。也可以依靠$\ln x \leq x-1(x>0) $,有 
\begin{equation}
    \begin{aligned}
        -D_{KL}(p||q)=\sum_{i=1}^{N}{p(x_i)\ln \frac{q(x_i)}{p(x_i)}}\leq \sum_{i=1}^{N}{p(x_i)(\frac{q(x_i)}{p(x_i)}-1) }\\ =\sum_{i=1}^{N}{q(x_i)-p(x_i)}=\sum_{i=1}^{N}{q(x_i)}-\sum_{i=1}^{N}{p(x_i)}=0
    \end{aligned}
    \nonumber
\end{equation}

\subsection{先验概率分布}
\par 
先验分布(prior distribution)一译“验前分布”“事前分布”。是概率分布的一种。与“后验分布”相对。与试验结果无关,或与随机抽样无关,反映在进行统计试验之前根据其他有关参数θ的知识而得到的分布。
贝叶斯学派认为,在进行观察以获得样本之前,人们对θ也会有一些知识。因为是在试验观察之前,故称之为先验知识。因此,贝叶斯派认为,应该把θ看作是随机变量。θ的分布函数记为H(θ),θ的密度函数记为h(θ),分别称为先验分布函数和先验密度函数,两者合称为先验分布。
\par
当参数 θ 的先验分布已知时,称在给定样本 x 下 θ 定条件分布为参数 θ 的后验分布(posterior distribution)。
后验分布可看成是在获得样本 x 后对参数先验知识的调整。
\par
根据原因来估计结果的概率分布即 似然估计。根据原因来统计各种可能结果的概率即似然函数。$${\mbox{后验概率}}=\frac{\mbox{似然概率}*\mbox{先验概率}}{evidence}$$
\subsection{隐变量的参数估计(MLE、MAP)问题}
\par
最大似然估计(ML,Maximum Likelihood)可以估计模型的参数。其目标是找出一组参数 $\theta$使得模型产生出观测数据 x 的概率最大:$$\arg\max\limits_{\theta}P(x|\theta)$$
\par
假如这个参数有一个先验概率,那么参数该怎么估计呢?这就是MAP要考虑的问题。 最大后验估计(MAP-Maxaposterior)。MAP优化的是一个后验概率,即给定了观测值后使概率最大:
$$\arg\max\limits_{\theta}P(x|\theta)=\arg\max\limits_{\theta}\frac{P(x|\theta)*P(\theta)}{P(x)}$$
\par
因为给定样本 x 后, p(x) 会在 θ空间上为一个定值,和 θ 的大小没有关系,所以可以省略分母p(x)。
$$\arg\max\limits_{\theta}P(x|\theta)=\arg\max\limits_{\theta}P(x|\theta)*P(\theta)$$即为
$$posterior\propto Likelihood*prior$$
P(x) 相当于是一个归一化项,整个公式就表示为:后验概率 正比于 先验概率*似然函数。
\section{自回归模型}
考虑markov过程
\par
PixelRNN和PixelCNN 通过链式法则计算似然函数$p(x)=\prod_{i=1}^{n}p(x_i|x_1,x_2,....,x_{i-1})$。然后最大化训练数据的似然概率。然而,生成的过程是序列处理的形式,尽管使用PixelCNN方法进行作为PixeRNN的改进,计算速度仍然很慢。
\par
PixelRNN 和 PixelCNN的优点是显式计算了似然概率$p(x)$

\section{EM算法}
\par
EM算法中文叫做:期望最大算法。主要解决的问题是:具有隐变量的混合模型的参数估计,即其极大似然估计。对于简单的模型$x\sim p(x)$我们可以直接计算p的参数$\theta$
$$\theta_{MLE}=\arg\max\limits_{\theta}\log p(x|\theta)$$ 

\section{白板推导三十节}
\subsection{生成模型定义}

% \maketitle
\par
对于给定的样本X,生成模型关注的是$\textbf{样本X的分布P(X)}$,生成模型和他要解决的任务无关,既可以解决监督任务,也可以解决非监督任务
\par
为什么生成模型是关注样本在样本空间的概率分布?考虑像素数28*28=784的MNIST的数据集,我们将其展开,则样本$\boldsymbol{x}=(x_1,x_2,\cdots,x_784)$,其中$x_i\in \{0,1\}$,那么可以认为$\{(x_1=x_{29}=\cdots=x_{757}=1),(x_2=x_{30}=\cdots=x_{758}=1),\cdots,(x_{28}=x_{56}=\cdots=x_{784}=1)\}$的样本是更像阿拉伯数字“1”的,在“1”的数据集出现的概率比较高。而其他样本如$(x_1=x_2=\cdots=x_{784}=0)$,就在数据集出现的次数很小,概率也小。
\par 
现在若给出了样本$X$的分布$P(X)$,我们从$p(x)$比较大的地方采样$x$,就可以生成模型
\par
对于前者,我们对标签Y一起建模求$P(X|Y)$
\subsection{无监督与有监督}
$$ \mbox{有监督学习}\left\{
\begin{aligned}
\mbox{数据:}&\mbox{\{(样本,标签)\}的有序对集合}\\
\mbox{目标:}&\mbox{学习x->y的映射} \\
\mbox{例子:}&\mbox{分类,回归,标记} 
\end{aligned}
\right.
$$

$$ \mbox{无监督学习}\left\{
\begin{aligned}
\mbox{数据:}&\mbox{\{(样本)\}的数据集合}\\
\mbox{目标:}&\mbox{学习样本中的模式与结构} \\
\mbox{例子:}&\mbox{降维,聚类,特征学习,密度估计,生成数据} 
\end{aligned}
\right.
$$

\par
我们认为:分类,回归,标记,为监督任务;而降维,聚类,特征学习,密度估计,生成数据为非监督任务
\par
$Naive\ Bayes 对于x\in\mathbb{R}^n\ P(X)= \prod_{i=1}^{n}P(X_i|y)$
\subsection{表示\&推断\&学习}
\subsection{模型分类}
\subsection{重参数技巧}

\section{变分自编码模型(Variational AutoEncoder)和变分推断}
\subsection{背景介绍}
在自编码器中 图像$x$经过encoder压缩转变成code空间的z,经过decoder输出$x'$。然而AE只是学到了一个$x$到$x'$的映射,并且在码空间中,数据的分布是离散的,码空间没有学采样函数,所以不能在码空间中虽以采样一个$z'$生成得到$x'$。我们希望在code空间中泛化
\par
\textbf{相对于AE,VAE预测了一个分布,再在分布中随机采样一个$z$,而噪声的分布(均值和方差)都是从样本中学习得到的,即从带噪声的code空间中采样}
\par
如果只像AE那样去训练重构图片和原始图片的差距${\left\| x-x'   \right\|}^2$会怎么样?则网络会把方差置0,模型退化为AE



\subsection{VAE就是无数的GMM}
GMM中,假设一共有m个高斯分布$(g_1,g_2,...g_m)$则x的分布P(x),可以用$$P(x)=\sum_{m}p(g_i)q(x|g_i)\qquad \sum_{m}p(g_i)=1$$,$p(g_i)$代表了属于哪个高斯分布的可能,$q(x|g_i)$中$x~N(\mu_i,\sigma_i)$为高斯分布 
\par
VAE可以认为是无数个GMM的叠加,是条件概率的积分$$P(x)=\int_z{p(z)q(x|z)}$$神经网络的输入是z,输出是对均值和方差的预测$\mu(z)$和$\sigma(z)$\textbf{VAE就是拿无限个高斯分布逼近真实的数据分布,隐变量z的每一维度代表了数据的一个属性},假定z为正态分布主要为了数学性质
\par
如果想极大似然$L=\int_x{\log P(x)}$来学习得到$P(x)=\int_z{p(z)q(x|z)}$的参数$\mu(z),\sigma(z)$这过于困难,因为$\mu(z),\sigma(z)$是网络的预测值,尽管我们能写出$P(z)$的分布,但是无法显示表达神经网络。高斯分布的$P(z)$中并不是都可以产生样本。我们需要另一个样本$q(z|x)$表现了哪些z产生的高斯分布,是对$P(x)$真正有效的,而其他的z就无需考虑。这实际上是一个编码器

\subsection{重参数技巧}
\par
采样时,为了得到隐变量z,我们不从
\section{生成对抗网络(2014)}
\par
讲故事.shwj虽然有100元大钞,但是他想靠画假钞来发财。刚开始有个新来的警察叔叔分不清假钞,shwj拿着自己乱画的100元造假钞是可以骗过他的,但是终究骗不过印钞机,于是shwj把真钞和假钞给了警察,让他做一些判断并纠正了他哪张是真钞,提升了判别假钞的能力.然后shwj回家好好学习画100元,企图让自己的画作骗过警察
\par
就这样几轮极限拉扯之后,警察叔叔的鉴别能力和shwj画假钞的能力一起提高了.最后shwj拿着真钞和假钞给警察叔叔看,但是此时警察叔叔没有了分辨能力,认为这些钞票是真钞的概率都是0.5,于是shwj就有了画出真实100元大钞的能力,
\par
设真实数据的采样为$x\sim p_{data}$,理想的数据满足分布$x\sim p_g$,先验的随机噪声分布为$z\sim p_z$,数据的生成器为$G(z;\theta_g)$,这里G代表了一个MLP构成的可微函数,另一个MLP为$D(X;\theta_d)$输出0或1:若x是从$p_g$中采样得到的,则输出1,反之输出0
\par
我们先把真实的数据和噪声生成的数据送入D中,希望D能尽可能判断出数据x的真实性,这个过程就是
$$\max\limits_{D}E_{x\sim P_{data}}[\log D(x)]+E_{z\sim p_z(z)}[\log(1-D(G(z)))]$$
\par
而对于判别器G,希望它能尽可能骗过G,让D(G(z))的值尽可能接近于1
$$\min\limits_{G}E_{z\sim P_{p_z(z)}}[\log(1-D(G(z)))]$$

\par 
总优化目标函数为$$\min\limits_{G}\max\limits_{D}V(D,G)=E_{x\sim P_{data}}[\log D(x)]+E_{z\sim p_z(z)}[\log(1-D(G(z)))]$$

\section{Flow-based Generative Model}
\par
invertible neural network 简单的数据分布到复杂的数据分布的映射
\par
生成模型的本质,就是希望用一个我们知道的概率模型来拟合所给的数据样本,也就是说,我们得写出一个带参数$\theta$的分布$q_{\theta}(x)$。然而,我们的神经网络只是“万能函数拟合器”,却不是“万能分布拟合器”,也就是它原则上能拟合任意函数,但不能随意拟合一个概率分布,因为概率分布有“非负”和“归一化”的要求。这样一来,我们能直接写出来的只有离散型的分布,或者是连续型的高斯分布。

\par
记生成器为$G$,网络定义了一个概率分布$p_G$ 对于$z\sim \pi(z)$输出$x=G(z)$是一个高维的向量,里面的每一个元素都是人脸图像的像素。
\par
我们希望G生成的照片足够真实,设真实的人脸照片$x$服从于分布$p_{data}(x)$,我们从其中采样m张图片,即选取$\{x^1,x^2,...,x^m\}from\ p_{data}(x)$
\par
求取$$G^*=\arg\max\limits_{g}{\sum_{i=1}^{m}{\log P_G(x^i)}}$$

\section{Diffusion Model}
\par

\subsection{Denoising Diffusion Probabilistic Model(DDPM)}
\par
DDPM有两个重要贡献。一个是DM想预测$x_t$直接预测$x_{t-1}$而DDPM认为其不便优化通过数学证明了可以以预测$x_{t-1}$到$x_{t-1}$所加的噪声达到同样的目的。第t步的时候Unet的输入除了有$x_t$,还有t的embedding(位置编码和傅里叶特征),这是因为采样过程中Unet的参数是每一步都共享的,希望开始的时候(t较大时)多设工程一些图像的论文,而到了t接近0时,多生成一些图像的如边角高频信息。这样模型就变成了
$$P(x_{t-1}|x_t)=\left\|   z_t-f_{\epsilon}(x_t,t_{embedding})  \right\|$$
\par
另一个贡献是DDPM 

前向过程中由于每个时刻$t$只与$t-1$时刻有关,所以可以看做马尔科夫过程,在马尔科夫链的前向采样过程中,也就是扩散过程中可以将数据转换为高斯分布。即扩散过程通过$T$次累积对输入数据$x_i$添加高斯噪声,将这个跟马尔可夫假设相结合,于是可以对扩散过程表达成
$$q(x_{1:T}|x_0):=  \prod_{t=1}^T{q(x_t|x_{t-1})}:= \prod_{t=1}^T{N(x_t; \sqrt{1-{\beta}_t}x_{t-1},{\beta}_t \boldsymbol{I})}$$ 
其中${\beta}_1,{\beta}_2,\cdots,{\beta}_T,$是高斯分布方差的超参数。在扩散过程中,随着$t$的增大,$x_t$越来越接近纯噪声。当$T$足够大的时候,$x_t$可以收敛为标准高斯噪声$N(0,\boldsymbol{I})$
\par
逆向过程时模型学习逆扩散过程的概率分布,以从纯高斯噪声$p(x_T):=N(x_T;0,\boldsymbol{I})$开始生成新数据。根据马尔可夫规则表示,逆扩散过程当前时间步$t$只取决于上一个时间步$t-1$,模型将逐步学习概率分布
$$p_{\theta}(x_{t-1}|x_t) :=N(x_{t-1};{\mu}_{\theta}(x_t,t),{\Sigma}_{\theta}(x_t,t))$$
得到联合概率分布$p_{\theta}(x_{T:0})$:
$$p_{\theta}(x_{T:0}):=p(x_T) \prod_{t=1}^T{p_{\theta}(x_{t-1}|x_t)} :=p_{\theta}(x_{T:0}):=p(x_T) \prod_{t=1}^T{N(x_{t-1};{\mu}_{\theta}(x_t,t),{\Sigma}_{\theta}(x_t,t))}$$
就可以从$p_\theta(x_0)$中采样得到$x_0$
这样一来,DDPM的每一步的推断可以总结为:
 
\begin{enumerate} 
    \item 每个时间步通过$x_t$和$t$来预测高斯噪声$z_{\theta}(x_t,t)$,随后根据${\mu}_{\theta}(x_t,t)=\frac{1}{\sqrt{a_t}}(x_t-\frac{{\beta}_t}{\sqrt{1-\overline{a_t}}})z_{\theta}(x_t,t)$得到均值${\mu}_{\theta}(x_t,t)$.
    \item 得到方差$\Sigma(x_t,t)= \widetilde{{\beta}_t}=\frac{1-\overline{{\alpha}_{t-1}}}{1-\overline{{\alpha}_t}}{\beta}_t$
    \item 利用$p_{\theta}(x_{t-1}|x_t) =N(x_{t-1};{\mu}_{\theta}(x_t,t),{\Sigma}_{\theta}(x_t,t))$得到$x_{t-1}$
\end{enumerate}

(https://zhuanlan.zhihu.com/p/572263021)





\subsection{Improved DDPM}
\par
DDPM认为噪声的方差是不用学习的,但是DDPM+认为学习到方差后生成的效果更优
\par
第二个改进是把添加噪声的schedule从linear变成了cosEmbedding,更work了
\par
第三个改进就是发现大模型对图像生成效果更优
\subsection{Diffusion Beats GAN}
\textcolor[rgb]{1,0,0}{classifier guidance对采样有很大帮助} ,classifier guided Diffusion是在训练模型的同时,再去训练一个分类器,因为DM的图片都是加了噪声的,所以这个classifier guided Diffusion是在加了噪声的ImageNet上训练的
\par
每次得到一个图片$x_t$后将图片输入到判别器中得到交叉熵损失g,g包含了图像是否有图像和图像是否真实到能被判别器判别的信息(这类似于GAN的Discriminator),然后计算$\nabla g$,得到下一步图像$x_{t-1}$生成的方向。牺牲了一部分图像的多样性换来图像的写实性
\par
classifer guided Diffusion\footnote{Scale Matters: Money is All your need}

\subsection{Denoising Diffusion Implicit Models(DDIM)}

\subsection{Score-based Model}
\par
思想来自于19年的论文《Generative Modeling by Estimating Gradients of the Data Distribution》,Gradients of the Data Distribution就是model的score。问题是如何求score,怎么用score作加强。作者提出的\emph{Langevin dynamics}实际上和DDPM的采样思路类似
\par
传统的生成模型都是学习数据分布的概率$p_{\theta}(x)$,其中x是样本。而score是对数概率的梯度,即$\nabla_x \log p_{\theta}(x)$,在样本空间中,梯度是指向样本密度最高的位置。如果我们获得了样本空间的每一点的score,那么对于生成的每一步,我们只需要沿着当位置的score方向更新数据,那么就会逼近真实的样本。
\par


\par
具体地,如果我们我们通过了某种方法(score matching)得到了score的模型$s_{\theta}(x) = \nabla_x \log p_{\theta}(x)$,那么我们就可以用Langevin dynamics的迭代过程从\textbf{一个任意分布走到目标分布}$$x_{i+1}\leftarrow x_i+\epsilon \nabla_x \log p_{\theta}(x)+\sqrt{2\epsilon}z_i, \qquad i=0,1,\cdots,K$$
在原始的数据上加噪声是为了更好地估计score。训练地时候只是训练了一个去噪模型,为什么可以做到图像生成?这是由于:
$$\mbox{估计噪声}\Leftrightarrow\mbox{估计score}\Leftrightarrow\mbox{估计数据分布梯度}$$
\par
而获得了梯度后,我们就可以像gradient descent一样去逐步逼近真实的数据分布,达到图像生成的效果
\subsection{背景介绍}
啊啊啊啊



% \end{sloppypar}
\end{document}